\documentclass[11pt]{article}

% --------------------
% Packages
% --------------------
\usepackage[margin=1in]{geometry}
\usepackage{amsmath, amssymb}
\usepackage{graphicx}
\usepackage{caption}
\usepackage{subcaption}
\usepackage{hyperref}
\usepackage{enumitem}
\usepackage{setspace}

\setlength{\parskip}{0.75em}
\setlength{\parindent}{0pt}

% --------------------
% Title Info
% --------------------
\title{\textbf{CubeSat Attitude Determination and Control System \\
Hardware-in-the-Loop Validation Using a 3-DOF Air Bearing Project Proposal}}
\author{Caleb Oblepias}
\date{\today}

% --------------------
% Document
% --------------------
\begin{document}

\maketitle

\vspace{-1em}

\section{Objective}
The objective of this project is to classify performance metrics and system objectives for a three-axis CubeSat Attitude Determination and Control System (ADCS). The system will be evaluated through a hardware-in-the-loop (HIL) testbed mounted on a three-degree-of-freedom air bearing representative of space conditions.

\section{System Overview}
A flight-representative CubeSat platform will be mounted on a 3-DOF air bearing
to enable near-frictionless rotation. Attitude actuation will be provided by a
set of reaction wheels, while onboard sensors provide state feedback. A microcontroller executes
the ADCS algorithms in real time, interfacing with both physical hardware and a dynamics simulator.

\section{ADCS Architecture}
Attitude estimation is performed using sensor fusion techniques to estimate
spacecraft attitude and angular rate. Control laws generate reaction wheel
torque commands to perform detumbling, orientation transition (slew maneuvers), and steady-state
pointing. The controller is designed to respect actuator limits while
maintaining stability and performance.

\section{Hardware-in-the-Loop Framework}
The HIL framework couples the physical spacecraft-representative hardware with a simulated orbital
environment. Reaction wheel torques are applied to the air-bearing platform,
while environmental effects and sensor noise are injected through simulation.
This enables validation of control performance and timing
under realistic operating conditions.

\section{Performance Metrics}
The system will be evaluated using quantitative metrics including:
\begin{itemize}[leftmargin=1.5em]
  \item Detumbling to angular rates below ? deg/s
  \item Detumbling time below ? s
  \item Steady-state pointing error below ? deg
  \item Slew maneuver settling time under ? s
  \item Real-time firmware performance
\end{itemize}

\section{Expected Outcomes}
This project will deliver a fully integrated CubeSat ADCS HIL testbed,
experimental validation of closed-loop real-time attitude control, quantified detumbling/pointing/slew  maneuver demonstrations, and documented
lessons learned applicable to flight CubeSat missions.

\vspace{1em}
\rule{\linewidth}{0.4pt}


\end{document}

